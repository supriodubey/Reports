\newpage
\begin{center}
    \huge \textbf{Abstract}
\end{center}

\vspace{0.5in}

One of the most sought unanswered questions in cosmology is what is Dark Matter? Dark Matter is thought to account for approximately $ 85\%$ of the matter contained in the universe and yet the exact nature of it is still unknown.  In 1971 it was proposed by Hawking\cite{1971MNRAS.152...75H} that a highly overdense region in the primordial universe could undergo gravitational collapse directly to form black holes.  Primordial Black Holes(PBH), are thought to be one of the potential candidates for dark matter as they form before matter-radiation dominated equality, hence they are non-baryonic. This makes them an intriguing subject for research. Large perturbations that exit the horizon during inflation may result in the formation of PBH. The recent gravitational waves (GWs) detection of tens of solar  masses black hole binaries by the LIGO-Virgo collaboration has stoked interest in PBH as a dark matter. \\
The literature on PBH is reviewed in this dissertation. A brief explanation of the observational evidence and the dynamics of dark matter are included in Chapter 1 along with several essential cosmological concepts. We analyze the flaws in the conventional cosmological model and how inflation addresses them in Chapter 2. We will also talk about how quantum fluctuations brought on by inflation can result in PBH formation. The development of PBH, its abundance, the shortcomings of the simplified model and a brief discussion of the inflationary theories used to generate it will all be covered in Chapter 3. The summary of the observational constraints on the PBH mass function will follow as our final point.