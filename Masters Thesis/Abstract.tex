\newpage
\begin{center}
    \huge \textbf{Abstract}
\end{center}

\vspace{0.5in}

Estimation of precise cosmological parameters has been a longstanding challenge in modern cosmology. Forthcoming cosmological surveys such as DESI, Euclid, SPHEREX, and the Roman space telescope will allow us to observe a large fraction of the sky and provide a very detailed map of the large-scale structure of the universe. The vast amount of data collected from these surveys will make small scales accessible with high precision. The small, non-linear scales will help shed more light on unresolved topics in cosmology, such as explaining cosmological tensions, examining the early universe, studying neutrinos, and so on.\\
In this work, we address such questions by applying a set of different estimators to the Quijote-png halo catalogues and Quijote-massive neutrinos halo catalogues at non-linear cosmological scales, up to \( k_{\text{max}} = 0.5 \, h \, \text{Mpc}^{-1} \) to find the best approach to constrain Primordial non-Gaussianity (PNG) which gives insight into the early universe and total neutrino mass \( M_{\nu} \). We use various summary statistics including the power spectrum, bispectrum, halo mass function, marked power spectrum, and marked modal bispectrum to train deep neural networks (NN) and perform likelihood-free inference of cosmological, PNG, and $M\nu$ parameters. We also look into a thorough comparison of summary statistics to determine their optimal combination in terms of PNG and $M\nu$ sensitivity.

.



