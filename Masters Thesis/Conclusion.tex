\chapter{Conclusion}

In this thesis, we developed a machine learning pipeline to perform Likelihood-Free Inference (LFI) for $\mathrm{M}_{\nu}$, $f_{NL}^{\mathrm{equil}}$, and other cosmological parameters using the moment network method. This approach avoids relying on assumptions or approximations of likelihood. Utilizing the Quijote and Quijote-PNG simulations, we identified various summary statistics and their combinations to constrain the sum of massive neutrinos $\mathrm{M}_{\nu}$, $f_{NL}^{\mathrm{equil}}$, and other cosmological parameters by extending the analysis to non-linear scales up to $k_{\mathrm{max}} = 0.5,h,\mathrm{Mpc}^{-1}$, where significant signals for both $\mathrm{M}_{\nu}$ and $f_{NL}^{\mathrm{equil}}$ are expected.

We extracted the power spectrum, modal bispectrum, marked power spectrum, marked modal bispectrum, and halo mass function (HMF) in redshift space from both Quijote and Quijote-PNG halo catalogues. These were used to examine constraints on $\mathrm{M}_{\nu}$ and $f_{NL}^{\mathrm{equil}}$, along with other cosmological parameters. Our analysis focused on the halo field, as halos are biased tracers of the underlying matter field and are sites of galaxy formation. This approach represents an initial step toward constraining cosmological parameters using actual observational data.

We found that incorporating marked statistics and the Halo Mass Function (HMF) can show promising potential for detecting  $f_{NL}^{\mathrm{equil}}$ signals and provide a moderate contribution to the study of $\mathrm{M}_{\nu}$ constraints from the upcoming large-scale structure (LSS) surveys. The best constraints were observed when adding the marked power spectrum and HMF to our analysis. While we experimented with different combinations of marks, we found that using the mark with the smallest smoothing scale, i.e., $R = 20,h^{-1},\mathrm{Mpc}$, $p = 1$, and $\delta{s} = 0.50$, provided constraints equivalent to the combination of all marks. Adding a marked bispectrum, however, did not yield any additional constraints.

Although our neural network (NN) model demonstrated good learning capabilities, as evidenced by the training vs. validation loss plots in Fig. \ref{fig:fnl_loss_plot} and Fig. \ref{fig:mnu_loss_plot}, the accuracy of our model predictions was low for $f_{NL}^{\mathrm{equil}}$ and $\mathrm{M}_{\nu}$ due to the limited dataset and restrictive prior boundaries.

In the future, we aim to test our model on the matter field for $\mathrm{M}_{\nu}$, which has been shown to provide more constraints as indicated in \cite{Massara_2021}. Additionally, we plan to perform a Fisher analysis of the combination of different statistics on massive neutrinos for the halo field to further validate our current findings.
