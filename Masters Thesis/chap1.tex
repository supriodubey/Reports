\chapter{Introduction}

The standard $\Lambda$CDM model is both a boon and a bane for our current understanding of the universe. It quite well explains the formation of large-scale structures, the state of the early universe, and the cosmic abundance of different types of matter and energy. On the other hand, despite the standard $\Lambda$CDM model's continued success in observations, there is growing interest in expanding cosmology beyond it because of some profound theoretical challenges, like the enigmatic physical origin of dark matter and dark energy and the apparent discrepancies between the observed and predicted clustering properties of cold dark matter (CDM) on small scales, which leave a large unanswered gap.

The standard model involves an early period where the density perturbations are very close to the Gaussian field with all pertinent information contained in the power spectrum, which is the Fourier transform of the two-point correlation function. The evolution of the universe to low redshifts resulted in these fluctuations becoming non-Gaussian on small scales owing to the nonlinear gravitational evolution. This leads to information leakage from the two-point function to higher-order statistics.

Neutrinos with finite mass suppress the small-scale power spectrum below the neutrino free-streaming scale. The suppression is proportional to the sum of the neutrino masses. This allows us to measure the total neutrino mass ($M_\nu$), determining which is one of the major challenges in cosmology as traditional power spectrum analysis is no longer applicable in the non-linear regime.

Another challenge is extracting primordial non-Gaussianity (PNG) information, especially in the case of equilateral shape, which we will focus on in this thesis from the non-Gaussianity produced by non-linear gravitational dynamics at low redshifts due to its comparatively weak signal. PNG is the deviation of the primordial perturbation distribution from Gaussianity. Improving the current constraint on the parameter $f_{\mathrm{NL}^{equil}}$ that characterizes the size of Equilateral PNG can help us develop a unique insight into the physics of the early universe.

The analysis of the large-scale structure of the Universe (LSS) at small, non-linear scales holds significant promise for such cosmological problems. Current and forthcoming cosmological surveys such as DESI, Euclid, SPHEREX, and the Roman space telescope will allow us to observe a large fraction of the sky and provide very detailed maps of the large-scale structure of the universe. The vast amounts of data collected from these surveys, i.e., all the small-scale modes, cannot be analyzed using fully analytical models hence leveraging into higher-statistical-significance measurements becomes imperative to extract meaningful insights.

In this work, we perform a thorough analysis of the dark matter halo field using various summary statistics and their combinations in Quijote N-body simulations \cite{Villaescusa_Navarro_2020}. We explore simulations with PNG initial conditions (Quijote-PNG) \cite{Coulton_2023} and simulations with massive neutrinos at non-linear cosmological scales, up to $k_{\mathrm{max}} = 0.5 \, h \, \mathrm{Mpc}^{-1}$, aiming to identify the most effective methods for constraining $f_{NL}^{\mathrm{equil}}$  and total neutrino mass $\mathrm{M}_{\nu}$ respectively.

We use neural networks (NN) for likelihood-free inference using the moment network method developed by \cite{jeffrey2020solvinghighdimensionalparameterinference} which is advantageous as it allows exploration across a broad range of parameter values without assuming a specific fiducial cosmology or relying on Gaussian likelihood assumptions. It directly maps the summary statistics to the final parameters, bypassing the need to evaluate covariance or assume a particular form for the likelihood function.

We compare different combinations of summary statistics such as the power spectrum, modal bispectrum, halo mass function, marked power spectrum and marked modal bispectrum over a wide range of parameters in our neural network to determine their optimal combination in terms of $f_{NL}^{\mathrm{equil}}$ and $\mathrm{M}_{\nu}$  parameters sensitivity and help break degeneracies of them and the cosmological parameters. This method represents some of the initial efforts in the direction of future applications to PNG and $\mathrm{M}_{\nu}$  parameter inference on real data from galaxy surveys which will require the need to use more realistic galaxy mocks.

The thesis is organized as follows. In Chapter \ref{introduction}, we start with a basic overview of cosmology and then move on to discuss the early universe and the inflationary scenario. This chapter also includes discussions on the dynamics during that epoch, as well as brief introductions to primordial non-Gaussianity and massive neutrinos. Moving to Chapter \ref{PT}, we discuss different cosmological perturbation theories, covering both linear and non-linear scales, alongside the insights from N-body simulations crucial for understanding density fluctuations, particularly at small, strongly non-linear scales. Chapter \ref{stats} explores the theory of summary statistics used in our analysis, examining their dependence on PNG conditions and massive neutrinos. In Chapter \ref{method}, we detail the dataset and methodology we employed, featuring the use of NN for analysis. Finally, Chapter \ref{result} presents the findings derived from our study.