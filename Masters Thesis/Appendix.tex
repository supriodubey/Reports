\begin{appendices}


%%%%%%%%%%%%%%%%%%%%%%%%%%%%%%%%%%%%%%%%%%%%%%%%%%%%%%%%%%%%%%%%%%%%%%%%%%%%%%%%%%%%
\chapter{Window Function}\label{Window function}
A window function is a mathematical tool which is introduced to treat the analytically random field as random fields are not differentiable.
The smallest comoving scale of the universe is that of the typical separation between neighbouring galaxies of the order of 1 Mpc.
To exclude scales smaller than $R\left(r<R\right.$ or $\left. k>R^{-1}\right)$ we can filter the density field with a window function. This can be done in $\mathrm{k}$-space or $\mathrm{x}$-space.
The filtering in $\mathrm{x}$-space is done by convolution. We introduce a (usually spherically symmetric) window function $W(\mathbf{r} ; R)$ such that
\begin{align}
   \int d^d r W(\mathrm{r} ; R)=1 \label{c1}
\end{align}

(normalization) and $W \sim 0$ for $|\mathbf{r}| \gg R$ and define the filtered density field
\begin{align}
    \delta(\mathrm{x} ; R) \equiv(\delta * W)(\mathrm{x}) \equiv \int d^d \mathrm{x}^{\prime} \delta\left(\mathrm{x}^{\prime}\right) W\left(\mathrm{x}-\mathrm{x}^{\prime}\right)
\end{align}

Here, the desired resolution is indicated by $R$, and the functions $\delta(\mathrm{x}; R)$ and $W(\mathrm{x}; R)$ are taken into consideration. We will refer to the following notation as $W(\mathrm{x}; R)$, keeping the scale $R$ implicit. We can now write simply $W(r)$ because we also assume that $W$ is spherically symmetric.

Denote the Fourier coefficients of $\delta(\mathrm{x} ; R)$ by $\delta_{\mathrm{k}}(R)$. We use the Fourier series for $\delta(\mathrm{x})$ and $\delta(\mathrm{x}; R)$, however, since $W(r)$ vanishes for big $r$, we can use $ \Tilde{W}(k)$ instead, which is a Fourier transform. Thus, a mixed form of the convolution theorem is required. Let's state it clearly:

\begin{align}
    \delta_{\mathbf{k}}(R) & =\frac{1}{V} \int_V d^d x \delta(\mathbf{x} ; R) e^{-i \mathbf{k} \cdot \mathbf{x}}=\frac{1}{V} \int_V d^d x d^d x^{\prime} \delta\left(\mathbf{x}^{\prime}\right) W\left(\mathbf{x}-\mathbf{x}^{\prime}\right) e^{-i \mathbf{k} \cdot \mathbf{x}} \\
    & =\frac{1}{V} \int_V d^d x^{\prime} \delta\left(\mathbf{x}^{\prime}\right) e^{-i \mathbf{k} \cdot \mathbf{x}^{\prime}} \int d^d r W(\mathbf{r}) e^{-i \mathbf{k} \cdot \mathbf{r}}=\Tilde{W}(\mathbf{k}) \delta_{\mathbf{k}},
\end{align}

where
\begin{align}
    \Tilde{W}(\mathbf{k})=\int d^d r W(\mathbf{r}) e^{-i \mathbf{k} \cdot \mathbf{r}}
\end{align}

is the Fourier transform of $W(\mathbf{r})$ With our normalization, $W(\mathrm{r})$ has dimension $1 / V$ and $\Tilde{W}(\mathbf{k})$ is dimensionless with $\Tilde{W}(\mathbf{k}=0)=1$. Since $W(\mathbf{r})=W(r)$ is spherically symmetric, so is $\Tilde{W}(\mathbf{k})= \Tilde{W}(k)$. Since $W(-\mathbf{r})=W(\mathbf{r}), \Tilde{W}(\mathbf{k})$ is real.
For the correlations of these filtered Fourier coefficients, we get
\begin{align}
    \left\langle\delta_{\mathbf{k}}^*(R) \delta_{\mathbf{k}^{\prime}}(R)\right\rangle=\Tilde{W}(\mathbf{k})^* \Tilde{W}\left(\mathbf{k}^{\prime}\right)\left\langle\delta_{\mathbf{k}^*} \delta_{\mathbf{k}^{\prime}}\right\rangle=\frac{1}{V} \delta_{\mathbf{k k}^{\prime}} \Tilde{W}(k)^2 P(k)
\end{align}
so the filtered power spectra are
\begin{align}
    \Tilde{W}(k)^2 P(k) \quad \text { and } \quad \Tilde{W}(k)^2 \mathcal{P}(k)
\end{align}

The filtered correlation function is
\begin{align}
    \xi(\mathbf{r} ; R) \equiv\langle\delta(\mathbf{x} ; R) \delta(\mathbf{x}-\mathbf{r} ; R)\rangle=\frac{1}{(2 \pi)^d} \int d^d k e^{i \mathbf{k} \cdot \mathbf{r}}  \Tilde{W}(k)^2 P(k)
\end{align}
and the variance of the filtered density field is
\begin{align}
    \sigma^2(R) \equiv\left\langle\delta(\mathrm{x} ; R)^2\right\rangle=\xi(0 ; \mid R)=\int_0^{\infty}  \Tilde{W}(k)^2 \mathcal{P}(k) \frac{d k}{k}
\end{align}

The simplest window function is the top-hat window function
\begin{align}
    W_T(\mathrm{r}) \equiv \frac{1}{V(R)} \Theta\left(1-\frac{r}{R}\right)\label{c10}
\end{align}

and $W_T(\mathbf{r})=0$ elsewhere, i.e., $\delta(\mathrm{x})$ is filtered by replacing it with its mean value within the distance $R$.Its Fourier transformation is
\begin{align}
      \Tilde{W}_{R}^{\text {T}}(k)=3 \frac{\sin (k R)-(k R) \cos (k R)}{k^{3} R^{3}},
\end{align}
Mathematically more convenient is the Gaussian window function
\begin{align}
    W_G(r) \equiv \frac{1}{V_G(R)} e^{-\frac{1}{2} r^2 / R^2}\label{c11}
\end{align}

Where
\begin{align}
    V_G(R) \equiv \int d^d r e^{-\frac{1}{2}|\mathbf{r}|^2 / R^2}
\end{align}

is the volume of $W_G$. The volume of a window function is defined as what $\int d^d r W(\mathrm{r})$ would be if $W$ were normalized so that $W(0)=1$, instead of the normalization we chose in \ref{c1}. For the top hat, this is the volume over which the filter averages; for others, a generalization of this.
The volume of $W_G$ is 
\begin{align}
    V_G(R)=(2 \pi)^{d / 2} R^d
\end{align}

\begin{align}
    \Tilde{W}_{R}^{\text {G}}(k)=e^{-\frac{1}{2}(k R)^2}
\end{align}


There are also other choices available apart from these two.

In principle, there could be some ambiguity on which field has to be smoothed out if the radiation field $\delta$ or the curvature field $\zeta$. However, in linear order, the two possibilities are equivalent, in fact, if we apply the smoothing procedure to Eq. \eqref{eq:2.30} we obtain

\begin{align}
    \delta_{R}(\mathbf{x})= & \int d^{3} y W_{R}(|\mathbf{x}-\mathbf{y}|) \delta(\mathbf{y}) \\
    = & \frac{2(1+w)}{5+3 w} \frac{1}{a^{2} H^{2}} \int d^{3} y W_{R}(|\mathbf{x}-\mathbf{y}|) \nabla_{\mathbf{y}}^{2} \zeta(\mathbf{y}) \\
    = & \frac{2(1+w)}{5+3 w} \frac{1}{a^{2} H^{2}} \int d^{3} y\left\{\zeta(\mathbf{y}) \nabla_{\mathbf{y}}^{2} W_{R}(|\mathbf{x}-\mathbf{y}|)\right. \\
    & \left.\quad+\nabla_{\mathbf{y}} \cdot\left[W_{R}(|\mathbf{x}-\mathbf{y}|) \nabla_{\mathbf{y}} \zeta(\mathbf{y})-\zeta(\mathbf{y}) \nabla_{\mathbf{y}} W_{R}(|\mathbf{x}-\mathbf{y}|)\right]\right\},
\end{align}

where we have used the relation $\psi \nabla^{2} \phi=\phi \nabla^{2} \psi+\nabla \cdot(\psi \nabla \phi-\phi \nabla \psi)$ between two generic scalar fields $\psi(\mathbf{x})$ and $\phi(\mathbf{x})$. The second term in the integrand is a surface contribution and vanishes using the divergence theorem, under the fairly general assumption that $W_{R}$ and its derivative vanish at large scales. Furthermore, at least for the two window functions in equations  \ref{c10} and \ref{c11}, we have that $\nabla_{\mathbf{y}}^{2} W_{R}=\nabla_{\mathbf{x}}^{2} W_{R}$, therefore the above equation reads as

\begin{align}
    \delta_{R}(\mathbf{x})=\frac{2(1+w)}{5+3 w} \frac{1}{a^{2} H^{2}} \int d^{3} y \zeta(\mathbf{y}) \nabla_{\mathbf{x}}^{2} W_{R}(|\mathbf{x}-\mathbf{y}|)=\frac{2(1+w)}{5+3 w} \frac{1}{a^{2} H^{2}} \nabla_{\mathbf{x}}^{2} \zeta_{R}(\mathbf{x}),
\end{align}

which is the smoothed version of Eq.\eqref{eq:3.30}, as we expected. This proves the equivalence between smoothing out the density or the curvature field at a linear level.

In the end, the complete relation between the smoothed density field and curvature perturbation reads as

\begin{align}
    \delta_{R}(\mathbf{k}, \tau)=-\frac{2(1+w)}{5+3 w} \frac{k^{2}}{a^{2} H^{2}} T(k, \tau) \Tilde{W}_{R}(k) D_{1}^{(+)}(\tau)\zeta(\mathbf{k}) =  \frac{2}{3}T(k, \tau) \Tilde{W}_{R}(k) D_{1}^{(+)}(\tau) \Phi(\mathbf{k}) 
\end{align}

% %%%%%%%%%%%%%%%%%%%%%%%%%%%%%%%%%%%%%%%%%%%%%%%%%%%%%%%%%%%%%%%%%%%%%%%%%
\chapter{Additional Figures}
\section{\texorpdfstring{$f_{NL}^{\mathrm{equil}}$ LH summary statistics combination plots}{fNL equil LH summary statistics combination plots}}\label{fnlappendix}

\newcommand{\folders}{%
    % Pk0_MP0_0_HMF,%
    Pk0_MP0_0,%
    Pk0_MP0,%
    Pk0_Bk0_MP0_0_MB0_HMF,%
    Pk0_Bk0_MP0_0_MB0,%
    Pk0_Bk0_MP0_0,%
    Pk0_Bk0_MP0,%
    Pk0_Bk0_HMF,%
    Pk0_Bk0%
}

\foreach \folder in \folders {%
    \edef\temp{%
        \noexpand\begin{figure}[htbp]%
        \noexpand\centering%
        \noexpand\includegraphics[width=0.9\textwidth]{fnl_results/\folder/combined_plot_\folder.png}%
        \noexpand\caption{Predictions of parameters for the best-performing network trained on the combination of \detokenize\expandafter{\folder} statistics, applied to the test set of the $f_{NL}^{equil}$ LH.}%
        \noexpand\label{fig:\folder}%
        \noexpand\end{figure}%
    }%
    \temp%
}
\newpage

%%%%%%%%%%%%%%%%%%%%%%%%%%%%%%%%%%%%%%%%%%%%%%%%%%%%%%%%
\section{\texorpdfstring{nwLH summary statistics combination plots}{nwLH summary statistics combination plots}}\label{nwlhappendix}

\newcommand{\foldersnwlh}{%
    Pk0_MP0,%
    Pk0_Bk0,%
    Pk0_MP0_0_HMF,%
    Pk0_MP0_0,%
    Pk0_Bk0_MP0,%
    Pk0_Bk0_HMF,%
    Pk0_Bk0_MP0_0_HMF,%
    Pk0_Bk0_MP0_0,%
    Pk0_Bk0_MP0_0_MB0_0_HMF,%
    Pk0_Bk0_MP0_0_MB0_0%
}

\foreach \folder in \foldersnwlh {%
    \edef\temp{%
        \noexpand\begin{figure}[htbp]%
        \noexpand\centering%
        \noexpand\includegraphics[width=0.9\textwidth]{nwlh_results/\folder/combined_plot_\folder.png}%
        \noexpand\caption{Predictions of parameters for the best-performing network trained on the combination of \detokenize\expandafter{\folder} statistics, applied to the test set of the nwLH.}%
        \noexpand\label{fig:\foldernwlh}%
        \noexpand\end{figure}%
    }%
    \temp%
}
\end{appendices}
%%%%%%%%%%%%%%%%%%%%%%%%%%%%%%%%%%%%%%%%%%%%%%%%%%%%%%%%%%%%%%%%%%%%%%%%%%%%%%%%%%%%
